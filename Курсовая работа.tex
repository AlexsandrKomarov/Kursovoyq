\documentclass[russian,utf8,nocolumnxxxi,nocolumnxxxii]{eskdtext}
\usepackage[T1,T2A]{fontenc} 
\usepackage[utf8]{inputenc}
\usepackage[english,ukrainian,russian]{babel}
\usepackage{amssymb,amsmath}
\usepackage[shortlabels]{enumitem}
\usepackage{tikz}
\usepackage{pgfplots}
\usepackage{siunitx}
\usepackage[american,cuteinductors,smartlabels]{circuitikz}
\usepackage[backend=biber]{biblatex}
\addbibresource{error_estimation_otchet.bib}
\usepackage[]{hyperref}
\hypersetup{colorlinks=true}
\usepackage{textcomp}
\newcommand{\No}{\textnumero}
\ESKDdepartment{Федеральное агентство по образованию}
\ESKDcompany{Санкт-Петербургский государственный электротехнический университет "ЛЭТИ"}
\ESKDtitle{Пояснительная записка к Курсовой работе}
\ESKDsignature{Вариант №10}
\ESKDauthor{Комаров А.В.}
\ESKDchecker{Прокшин А.Н.}
\ESKDdocName{по дисциплине "Информатика"}
\begin{document}
\maketitle

\newpage
\tableofcontents

\newpage
\section{Введение}
Применение прикладных пакетов программ для выполнения математических расчетов при изучении высшей математики, а так же ряда специальных дисциплин является актуальным, особенно при решении задач профессиональной направленности, вычислительная часть решения которых обычно достаточно громоздка. В этом случае использование прикладных программ освобождает от рутинных вычислений и позволяет преподавателю больше времени уделить анализу условия задачи, рассмотреть различные методы и способы ее решения и провести анализ результатов.
\newpage
\section{Тема и цель курсовой работы}
\textbf{Тема курсовой работы}:Решение математиечских задач с использованием математических пакетов.

\textbf{Цель курсовой работы}: Научиться применять "Scilab" и "SMath Studio" и другие математические пакеты в различных отраслях инженерной деятельности.

\textbf{Задание к курсовой работе}:
\\1.Даны функции $f(x)=\sqrt{3}sin(x)+cos(x)$;$g(x)=cos(2x+\frac{\pi}{3})-1$;
\\Для них:
\begin{enumerate}
    \item[а)]Решить уравнение $f(x)=g(x)$
    \item[б)]Исследовать функцию $h(x)=f(x)-g(x)$ на промежутке $[0;\frac{5\pi}{6}]$
\end{enumerate}
\\2.Найти коэффициенты кубического сплайна, интерполирующего данные, представленные в векторах:\\
$V_{x}=[0;0,5;1,4;2,25;3,5]$,
$V_{y}=[6,0;5,7;6,875;6,333;5,167]$\\
Оценить погрешность интерполяции в точке $x=2,4$.Вычислить значение функции в точке $x=1,4$.Построить на графике функции f(x), полученную после нахождения коэффициентов кубического сплайна.\\
3.Решить задачу оптимального распределения неоднородных ресурсов.\\
Для изготовления $n$ видов изделий $N_1,N_2,...N_n$ необходимы ресурсы $m$ видов: трудовые, материальные, финансовые, и др. Известно требуемое количество $i-$го ресурса, которым предприятие располагает в данный момент, $-a_i$. Известна прибыль $P_i$, получаемая предприятием от изготовления каждого $j-$го изделия. Требуется определить, какие изделия и в каком количестве должны производиться предприятием, чтобы прибыль была максимальной.

\newpage
\section{Исследование функции}
\begin{enumerate}
    \item[a)]Решить уравнение $f(x)=g(x)$\\
    Так как $f(x)=g(x)$,то приравниваем выражение к нулю, и получаем:\\ $\sqrt{3}sin(x)+cos(x)-(2x+\frac{\pi}{3})-1=0$\\
    Для того,чтобы решить данное уравнение воспользуемся математическим пакетом "Scilab", из этого следует алгоритм решения:\\
    f(x)=g(x);{В h(x) = f (x) - g(x) х равен (Нахождение области определения функции )[х не равен]}:\\
    deff('[y]=h(x)','y1 = (sqrt(3))*(sin(x))+(cos(x)), y2 = cos((2*x) + ((pi)/3)) - 1, y=y1-y2')\\
    fsolve(0,h)\\
    Ответ: x=0.5235988
\end{enumerate}
\begin{enumerate}
    \item[б)]Исследовать функцию $h(x)=f(x)-g(x)$ на промежутке $[0;\frac{5\pi}{6}]$.
\end{enumerate}
\begin{center}
    \textbf{Алгоритм исследования функции.}
\end{center}
\begin{enumerate}
    \item[1)]Область определения функции.\\
    Поскольку области определения $sin(x)$ и $cos(x)$ являются множеством действительных числел, то для функции\\
    $h(x)=\sqrt{3}sin(x)+cos(x)-cos(2x+\frac{\pi}{3})-1$ областью определения является бесконечность, из этого также следует,что точки разрыва отсутствуют из-за неограниченности данной функции.
    \item[2)]Проверяем функцию на четность или нечетность.\\
    Для того, чтобы проверить функцию на четность или нечетность подставим $h(-x)$ вместо $h(x)$ и получим:\\
    $h(-x)=\sqrt{3}sin(-x)+cos(-x)-cos(2(-x)+\frac{\pi}{3})-1=-\sqrt{3}sin(x)+cos(x)-cos(-2x+\frac{\pi}{3})-1$;\\
    Из ответа следует,что функция поменяла знаки,следовательно, она гарантировано не является четной.\\
    Чтобы проверить является ли данная функция нечетной, перед получившейся функцией подставим знак минус и получим:\\
    $h(-x)=-\sqrt{3}sin(x)+cos(x)-cos(-2x+\frac{\pi}{3})-1=-(\sqrt{3}sin(x)-cos(x)+cos(-2x+\frac{\pi}{3})+1)$;\\
    Из этого следует,что данная функция не является четной и не является нечетной.
    \item[3)]Находим точки пересечения графика с осями координат.\\
    Находим нули функции-это точки пересечения графика функции\\
    $h=f(x)$ с осью абсцисс.\\
    Находим точку пересечения с осью $Ox$, приравнивая данную функцию к $0$.\\
    $h(x)=\sqrt{3}sin(x)+cos(x)-cos(2x+\frac{\pi}{3})-1$, при $x=0$\\
    $h(0)=\sqrt{3}sin(0)+cos(0)-cos(2*0+\frac{\pi}{3})-1=0$\\
    $x=\frac{5\pi}{6}$, $(\frac{5\pi}{6};0)$\\
    Находим точку пересечения с осью $Oy$, приравнивая данную функцию к $0$.\\
    $h(x)=\sqrt{3}sin(x)+cos(x)-cos(2x+\frac{\pi}{3})-1$, при $x=0$\\
    $h(x)=\sqrt{3}sin(x)+cos(x)-cos(2x+\frac{\pi}{3})-1=0$\\
    $x=1.5$, $(0;1.5)$\\
    
    \begin{figure}[h]
    \centering
        \begin{tikzpicture}
\draw node[below] {$0$};
\draw[->] (0,-0.1) -- (0,5) node[left] {$Y$};
\draw[->] (-0.1,0) -- (5,0) node[below] {$X$};
\draw [domain=0:2.6179938,thick,smooth,black] plot ({\x},{sqrt(3)*sin(\x r)+cos(\x r)-cos((2*\x r)+(pi/3 r))+1});
\draw (2.6179938,0) node[below] {$\frac{5\pi}{6}$};
\draw [dashed]  (0,4) node[left] {4} -- (1.047,4) -- (1.047,0) node[below] {1,047};
        \end{tikzpicture}
        \caption{График функции на промежутке $[0;\frac{5\pi}{6}]$}
    \end{figure}
    
    \item[4)]Исследуем функцию с помощью производной.\\
    Находим промежутки убывания и возрастания функции, а также точки минимума и максимума, для этого мы следуем определенному алгоритму:\\
    а) Находим промежутки убывания и возрастания функции:\\
    $h'(x)=\sqrt{3}sin(x)+cos(x)-cos(2x+\frac{\pi}{3})-1=-sin(x)+2sin(\frac{1}{3}(6x++\pi))+\sqrt{3}cos(x)$\\
    Теперь упростим и приравняем к 0,чтобы найти стационарные точки:\\
    $h'(x)=2sin(2x+\frac{\pi}{3})+2cos(x+\frac{\pi}{6})$\\
    $2sin(2x+\frac{\pi}{3})+2cos(x+\frac{\pi}{6})=0$\\
    \begin{figure}[h]
    \centering
    \begin{tikzpicture}
    \draw[->] (0,-0.1) -- (0,5) node[left] {$y$};
    \draw[->] (-0.1,0) -- (5,0) node[below] {$x$};
    \draw[domain=0:2.6179938,thick,smooth,black] plot ({\x},{2*sin((2*\x r)+(pi/3 r))+2*cos((\x r)+pi/6 r)});
    \draw [dashed] (2.6179938,-2) -- (2.6179938,0) node[below right] {$\frac{5\pi}{6}$};
    \draw (1.047,0) node{\textbullet};
    \draw (1.047,0) node[below left] {1,047};
    \draw (0,3.464) node[left] {3,464};
    \end{tikzpicture}
    \caption{Первая производная на промежутке $[0;\frac{5\pi}{6}]$}
    \end{figure}
    Из графика понимаем, что функция имеет одну стационарную точку $1,047$, на интервале $(0,1.047)$ производная положительная(функция возрастает), на интервале $(1.047,\frac{5\pi}{6})$ производная отрицательная(функция убывает).\\
    б)Находим точки перегиба и определяем выпуклость/вогнутость функции:\\
    $h''(x)=-\sqrt{3}sin(x)-cos(x)+4cos(2x+\frac{\pi}{3})=-2sin(x+\frac{\pi}{6})+4cos(2x+\frac{\pi}{3})$\\
    Приравниваем получившуюся функцию к 0 и получаем:\\
    $-2sin(x+\frac{\pi}{6})+4cos(2x+\frac{\pi}{3})=0$\\
    
    \begin{figure}[h]
        \centering
        \begin{tikzpicture}
        \draw[->] (0,0) -- (0,5) node[left] {$y$};
        \draw[->] (0,0) -- (5,0) node[below] {$x$};
        \draw (0,1) node[left] {1};
        \draw [domain=0:2.6179938,thick,smooth,black] plot ({\x},{-2*sin((\x r)+(pi/6 r))+4*cos((2*\x r)+(pi/3 r))});
        \draw (0.1113,0) node{\textbullet};
        \draw (0.1113,0) node[below left] {0,1113};
        \draw[dashed] (1.047,-6) -- (1.047,0) node[above] {$\frac{\pi}{3}$};
        \draw[dashed] (2.6179938,4) -- (2.6179938,0) node[below] {$\frac{5\pi}{6}$};
        \end{tikzpicture}
        \caption{Вторая производная на промежутке $[0;\frac{5\pi}{6}]$}
    \end{figure}
    \newpage
    На получившемся графике видно, что на интервале $(0.1113,1.983)$ вторая производная отрицательная,следовательно,на интервале\\ $(0.1113,1.983)$ функция выпуклая, а на остальных двух промежутках функция вогнута.
\end{enumerate}

\newpage
\section{Исследование кубического сплайна}
Найти коэффициенты кубического сплайна, интерполирующего данные, представленные в векторах:\\
$V_{x}=[0;0,5;1,4;2,25;3,5]$,
$V_{y}=[6,0;5,7;6,875;6,333;5,167]$\\
Оценить погрешность интерполяции в точке $x=2,4$.Вычислить значение функции в точке $x=1,4$.Построить на графике функции f(x), полученную после нахождения коэффициентов кубического сплайна.\\
\indent Интервал интерполяции разбивается на небольшие отрезки, на каждом из которых функция задается полиномом третьей степени.Коэффициенты полинома подбираются таким образом, чтобы выполнялись определенные условия.Общие для всех типов сплайнов третьего порядка требования - непрерывность функции, её первой и второй производных и прохождение ей через предписанные ей точки.Кубический сплайн задается значениями функции в узлах и значениями произвоных на границе отрезка интерполяции(либо первых, либо вторых производных).Формула кубического сплайна на каждом из частных отрезков $[x_i,x_{i+1}]$ имеет вид:\\
\begin{center}
    $S_{i}(x)=a_{i}+b_{i}(x-x_{i}+c_{i}(x-x_{i})^2+d_{i}(x-x_{i})^3$, 
\end{center}
где $a_{i},b_{i},c_{i},d_{i}$ - неизвестные.\\
Даны точки:\\
$G_{I}(0,6)$\\
$G_{II}(0.5,5.7)$\\
$G_{III}(1.4,6,875)$\\
$G_{IV}(2.25,6.333)$\\
$G_{V}(3.5,5.167)$\\

Найдем коэффициенты $А_i_j$, исходя из того, что в точках склейки функция не имеет разрывов, изломов и изгиб её слева и справа совпадает Записывая равенства через коэффициенты получаем 8 уравнений:\\
\begin{center}
$f_{1}(G_{I})=A_{10}+A_{11}G_{I}+A_{12}G_{I}^2+A_{13}G_{I}^3$\\
$f_{1}(G_{II})=A_{10}+A_{11}G_{II}+A_{12}G_{II}^2+A_{13}G_{II}^3$\\
$f_{2}(G_{II})=A_{20}+A_{21}G_{II}+A_{22}G_{II}^2+A_{23}G_{II}^3$\\
$f_{2}(G_{III})=A_{20}+A_{21}G_{III}+A_{22}G_{III}^2+A_{23}G_{III}^3$\\
$f_{3}(G_{III})=A_{30}+A_{31}G_{III}+A_{32}G_{III}^2+A_{33}G_{III}^3$\\
$f_{3}(G_{IV})=A_{30}+A_{31}G_{IV}+A_{32}G_{IV}^2+A_{33}G_{IV}^3$\\
$f_{4}(G_{IV})=A_{40}+A_{41}G_{IV}+A_{42}G_{IV}^2+A_{43}G_{IV}^3$\\
$f_{4}(G_{V})=A_{40}+A_{41}G_{V}+A_{42}G_{V}^2+A_{43}G_{V}^3$\\
\end{center}
Уравнения первой производной в точках склейки:\\
\begin{center}
$A_{11}+2A_{12}G_{II}+3A_{13}G_{II}^2=A_{21}+2A_{22}G_{II}+3A_{23}G_{II}^2$\\
$A_{21}+2A_{22}G_{III}+3A_{23}G_{III}^2=A_{31}+A_{32}G_{III}+3A_{33}G_{III}^2$\\
$A_{31}+2A_{32}G_{IV}+3A_{33}G_{IV}^2=A_{41}+A_{42}G_{IV}+3A_{43}G_{IV}^2$\\
\end{center}
Уравнения второй производной в точках склейки:\\
\begin{center}
$2A_{12}+6A_{13}G_{II}=2A_{22}+6A_{23}G_{II}$\\
$2A_{22}+6A_{23}G_{III}=2A_{32}+6A_{33}G_{III}$\\
$2A_{32}+6A_{33}G_{IV}=2A_{42}+6A_{42}+6A_{43}G_{IV}$\\
Ещё два уравнения получаем из граничных условий в крайних точках:\\
$2A_{12}+6A_{13}G_{I}=0$\\
$2A_{42}+6A_{43}G_{V}=0$\\
\end{center}
Из получившихся уравнений составим матрицу:\\

\makeatletter
\renewcommand*\env@matrix[1][c]{\hskip -\arraycolsep
\let\@ifnextchar\new@ifnextchar
\array{*\c@MaxMatrixCols #1}}
\makeatother

\resizebox{13cm}{!}{\begin{pmatrix}[0.0001cm]
$ 1& G_{I}& G_{I}^2& G_{I}^3& 0& 0& 0& 0& 0& 0& 0& 0& 0& 0& 0& 0$\\
$ 1& G_{II}& G_{II}^2& G_{III}& 0& 0& 0& 0& 0& 0& 0& 0& 0& 0& 0& 0$\\
$ 0& 1& 2G_{II}& 3G_{II}^2& 0& -1& -2G_{II}& -3G_{II}^2& 0& 0& 0& 0& 0& 0& 0& 0$\\
$ 0& 0& 2& 6G_{II}& 0& 0& -2& -6G_{II}& 0& 0& 0& 0& 0& 0& 0& 0$\\
$ 0& 0& 0& 0& 1& G_{II}& G_{II}^2& G_{II}^3& 0& 0& 0& 0& 0& 0& 0& 0$\\
$ 0& 0& 0& 0& 1& G_{III}& G_{III}^2& G_{III}^3& 0& 0& 0& 0& 0& 0& 0& 0$\\
$ 0& 0& 0& 0& 0& 1& 2G_{III}& 3G_{III}^2& 0& -1& -2G_{III}& -3G_{III}^2& 0& 0& 0& 0$\\
$ 0& 0& 0& 0& 0& 0& 2& 6G_{III}& 0& 0& -2& -6G_{III}& 0& 0& 0& 0$\\
$ 0& 0& 0& 0& 0& 0& 0& 0& 1& G_{III}& G_{III}^2& G_{III}^3& 0& 0& 0& 0$\\
$ 0& 0& 0& 0& 0& 0& 0& 0& 1& G_{IV}& G_{IV}^2& G_{IV}^3& 0& 0& 0& 0$\\
$ 0& 0& 0& 0& 0& 0& 0& 0& 0& 1& 2G_{IV}& 3G_{IV}^2& 0& -1& -2G_{IV}& -3G_{IV}^2$\\
$ 0& 0& 0& 0& 0& 0& 0& 0& 0& 0& 2& 6G_{IV}& 0& 0& -2& -6G_{IV}$\\
$ 0& 0& 0& 0& 0& 0& 0& 0& 0& 0& 0& 0& 1& G_{IV}& G_{IV}^2& G_{IV}^3$\\
$ 0& 0& 0& 0& 0& 0& 0& 0& 0& 0& 0& 0& 1& G_{V}& G_{V}^2& G_{V}^3$\\
$ 0& 0& 2& 6G_{I}& 0& 0& 0& 0& 0& 0& 0& 0& 0& 0& 0& 0$\\
$ 0& 0& 0& 0& 0& 0& 0& 0& 0& 0& 0& 0& 0& 0& 2& 6G_{V}$
\end{pmatrix}}*\resizebox{1.14cm}{!}{\begin{pmatrix}
$A_{10}$\\
$A_{11}$\\
$A_{12}$\\
$A_{13}$\\
$A_{20}$\\
$A_{21}$\\
$A_{22}$\\
$A_{23}$\\
$A_{30}$\\
$A_{31}$\\
$A_{32}$\\
$A_{33}$\\
$A_{40}$\\
$A_{41}$\\
$A_{42}$\\
$A_{43}$
\end{pmatrix}}=\resizebox{1.06cm}{!}{\begin{pmatrix}
G_{I}\\
G_{II}\\
0\\
0\\
G_{II}\\
G_{III}\\
0\\
0\\
G_{III}\\
G_{IV}\\
0\\
0\\
G_{IV}\\
G_{V}\\
0\\
0
\end{pmatrix}}\\

Ответ:
\resizebox{1.7cm}{!}{\begin{pmatrix}
6\\
-1.0723\\
0\\
1.8892\\
6.4814\\
-3.9608\\
5.777\\
-1.9622\\
-1.8643\\
13.9228\\
-6.997\\
1.0792\\
11.3034\\
-3.6341\\
0.8061\\
-0.0768
\end{pmatrix}}\\
\\
Уравнения для сплайна:\\
\begin{cases}
$f_1(x)=1.8892x^3-1.0723x+6$\\
$f_2(x)=-1.9622x^3+5.777x^2-3.9608x+6.4814$\\
$f_3(x)=1.0792x^3-6.997x^2+13.9228x-1.8643$\\
$f_4(x)=-0.0768x^3+0.8061x^2-3.6341x+11.3034$\\
\end{cases}
\begin{figure}[h]
    \centering
    \begin{tikzpicture}[xscale=4.5]
    \draw[<->] (0,7.0) node[left] {$y$} -- (0,0) node[below] {$0$} -- (3.5,0) node[below] {$x$};
    \draw[domain=0:0.5,smooth, black] plot ({\x},{(1.8892*((\x)*(\x)*(\x)))-(1.0723*(\x))+6});
    \draw[domain=0.5:1.4,smooth,green] plot ({\x},{(-1.9622*((\x)*(\x)*(\x)))+(5.777*((\x)*(\x)))-(3.9608*(\x))+6.4814});
    \draw[domain=1.4:2.25,smooth, blue] plot ({\x},{(1.0792*((\x)*(\x)*(\x)))-(6.997*((\x)*(\x)))+(13.9228*(\x))-1.8643});
    \draw[domain=2.25:3.5,smooth, red] plot ({\x},{(-0.0768*((\x)*(\x)*(\x)))+(0.8061*((\x)*(\x)))-(3.6341*(\x))+11.3034});
    \draw (0,6) node{\textbullet} node[left] {(0,6)};
    \draw (0.5,5.7) node{\textbullet} node[below] {(0.5,5.7)};
    \draw (1.4,6.875) node{\textbullet} node[above] {(1.4,6.875)};
    \draw (2.25,6.333) node{\textbullet} node[below left] {(2.25,6,333)};
    \draw (3.5,5.167) node{\textbullet} node[above] {(3.5,5.167)};
    \end{tikzpicture}
    \caption{Сплайн}
\end{figure}


\newpage
\begin{center}
    \textbf{Расчет погрешности интерполяции про помощи этмитовых кубических сплайнов.}
\end{center}
Если функция достаточно гладкая, то:\\
$$|S_3^{(r)}(x)-f^{(r)}(x)|\leqslant \frac{1}{384}\overline{h}^4|f^{IV}(x)|$$\\
Так как нам известна функция $f(x)$, с которой были взяты координаты,следовательно, для оценки погрешности необходимо взять $f''''(x)$ от неизвестной функции.Для нахождения данной производной нужно воспользоваться полиномом Ньютона:\\
$N(x)=A_0+A_1(x-x_0)+A_2(x-x_0)(x-x_1)+A_3(x-x_0)(x-x_1)(x-x_2)+A_4(x-x_0)(x-x_1)(x-x_2)(x-x_3)$\\
Для вычисления коэффициентов $A_1,A_2,A_3,A_4$ воспользуемся формулой разделенной разности:\\
\begin{center}
$A_0=f(x_0)$\\
$A_1=\frac{f(x_1)}{x_1-x_0}+\frac{f(x_0)}{x_0-x_1}$\\
$A_2=\frac{f(x_2)}{(x_2-x_1)(x_2-x_0)}+\frac{f(x_1)}{(x_1-x_2)(x_1-x_0)}+\frac{f(x_0)}{(x_0-x_2)(x_0-x_1)}$\\
$A_3=\frac{f(x_3)}{(x_3-x_2)(x_3-x_1)(x_3-x_0)}+\frac{f(x_2)}{(x_2-x_0)(x_2-x_1)(x_2-x_3)}+\frac{f(x_1)}{(x_1-x_0)(x_1-x_2)(x_1-x_3)}+\frac{f(x_0)}{(x_0-x_1)(x_0-x_2)(x_0-x_3)}$\\
$A_4=\frac{f(x_4)}{(x_4-x_3)(x_4-x_2)(x_4-x_1)(x_4-x_0)}+\frac{f(x_3)}{(x_3-x_4)(x_3-x_2)(x_3-x_1)}+\frac{f(x_2)}{(x_2-x_3)(x_2-x_4)(x_2-x_1)(x_2-x_0)}+\frac{f(x_1)}{(x_1-x_3)(x_1-x_2)(x_1-x_4)(x_1-x_0)}+\frac{f(x_0)}{(x_0-x_3)(x_0-x_2)(x_0-x_1)(x_0-x_4)}$
\end{center}
Коэффициенты равны: $A_0=6$,$A_1=-0.6$,$A_2=-56$,$A_3=-1.09845$,$A_4=0.4062098$\\
Переменная $h$ является ближайшей табличной координатой и координатой просчитываемой точки погрешности\\
Получившиеся значения подставляем в полином Ньютона и получаем:\\
$N(x)=-256.07297$\\
Из этого расчета, с погрешностью в точке $x=2.4$, получаем погрешность равную $3.8138*10^{-4}$

\newpage
\section{Решение задач оптимизации}
Составим расчетную программу на языке программирования "Packal ABC",\\
она имеет схожий синтакис с математическим пакетом "Scilab".\\

program task3;\\
var i1, i2, i3, i4: array [1..4] of integer;\\
k1, k2, k3, k4, max, i, j, t, m, f:integer;\\
begin\\
max := 0;\\
i1[1] := 4;\\
i1[2] := 4;\\
i1[3] := 6;\\
i1[4] := 40;\\
i2[1] := 4;\\
i2[2] := 6;\\
i2[3] := 4;\\
i2[4] := 55;\\
i3[1] := 4;\\
i3[2] := 6;\\
i3[3] := 5;\\
i3[4] := 35;\\
i4[1] := 6;\\
i4[2] := 3;\\
i4[3] := 8;\\
i4[4] := 25;\\
t := 14;\\
m := 12;\\
f := 35;\\
for i := 0 to 4 do\\
for j := 0 to 4 do\\
if (((t - i1[1]*j - i2[1]*j-i3[1]*j-i4[1]*j) >= 0) and ((m - i1[2]*i - i2[2]*j-i3[2]*j-i4[2]*j) >= 0) and ((f - i1[3]*i - i2[3]*j-i3[3]*j-i4[3]*j) >= 0)) then\\
if ((i1[4]*i + i2[4]*j+i3[4]*j+i4[4]*j) > max) then\\
begin\\
max := i1[4]*i + i2[4]*j+i3[4]*j+i4[4]*j;\\
k3 := k1;\\
k4 := k2;\\
k1 := i;\\
k2 := j;\\
end;\\
write(max, ' ', k1, ' ', k2, ' ', k3, ' ', k4);\\
end.\\
Ответ: 120 3 0 2 0\\
Для достижения максимальной прибыли 120, три единицы первого изделия и две единицы второго изделия.\\
\newpage
\section{Вывод}
В данной курсовой работе было изучено исследование функции, построение сплайнов с помощью математического пакета «SciLab». Были изучены возможности взятия производной такими командами как diff, позволяющая брать дифференциалы и numdiff позволяющая находить производную на отдельном промежутке чисел. В работе было изучено построение графиков в данном математическом пакете с помощью функции задания функций function и команды для построения графика plot.  Так же оказалось, что пакет содержит возможности оператора if, который был использован для определения чётности и не чётности функции. Были изучены особенности задачи матриц и матричных уравнений в математическом пакете  «SMath Studio».
\newpage
\section{Список используемой литературы}
\begin{enumerate}
  \item[1]Ю.С. Завьялов. Методы сплайн-функций. М.Наука, 1980.
  \item[2]Калиткин. Численные методы. М.,Мир, 1980.
  \item[3]Чеснокова. Рудченкоюю Решение инженерных и математических задач. М. «БИНОМ. Лаборатория знаний». 2008.
  \item[4]Тропин. Михайлова. Михайлов. Численные и технические расчеты в среде Scilab
(ПО для решения задач
численных и технических вычислений).  М.,Наука, 1980.
\item[5]Андриевский А.Б., Андриевский Б.Р., Капитонов А.А., Фрадков А.Л. \textit{РЕШЕНИЕ ИНЖЕНЕРНЫХ ЗАДАЧ В SCILAB} - Санкт-Петербург: НИУ ИТМО, 2013. - 97 с.
\end{enumerate}
\end{document}